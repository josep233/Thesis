\chapter{Finite Element Model Automation and Validation }
\label{ch:ch2}

The 3D parameterized maize stalk model allows for complete control of all model parameters. The method by which these models are structurally analyzed is finite element analysis (FEA). FEA operates by dividing a geometry into small elements, which discretizes the problem into a system of equations that can be solved iteratively. In order to solve this system of equations, a user needs to define many finite element program parameters.

Creating a finite element model in any situation is an involved process: the researcher must manually define all necessary material constants, boundary conditions, and meshes (dividing up the geometry into small elements) to receive structural responses. Even when a researcher gets FEA results, a great deal of post processing is often necessary.

Setting up and processing finite element models is time-consuming, especially when thousands of models are required. To address this, the FEA process for the 3D parameterized maize stalk model was automated. This chapter details the automation process, resulting in a package that will be used in future studies.

\section{Stalk Geometry Generation}
\label{sec:stalk_geometry_generation}
This chapter explains how the structural responses of maize stalk geometries were calculated. This chapter does \textit{not} address how the maize stalk geometries were generated. Instead, we refer to a previous study by Ottesen, who developed the geometry generation approach~\scite{ottesen_parameterised_2022,ottesen_development_2023}. Ottesen’s approach was used to generate two solid geometries for each stalk: one for the pith (the inner stalk), and one for the rind (the outer stalk). These geometries were generated in SolidWorks .STEP format (Dassault Systèmes SE, Vélizy-Villacoublay, France). Ottesen’s geometry generation technique functions separately from the automation technique outlined in this chapter. As such, all geometries needed to be generated before any finite element operations were applied to them.
\newline
\newline

\section{The Automation System}
\label{sec:the_automation_system}
Automation was implemented using the procedural programming paradigm. Procedural programming achieves an objective through a series of steps called functions~\scite{blokdyk_procedural_2020}. Each function has a specific role in computing a global result. This allows a programmer to write only a few functions that are reused for every new set of inputs, without having to rewrite code for a new situation.

Procedural programming was applied to this problem by creating a list of functions that could create a maize stalk geometry, import the geometry into Abaqus 2022 (Systemes Simulia Corporation, Providence, RI, USA), apply all conditions necessary for analysis, run an analysis, and then extract analysis results. These functions were written in Python and executed via the Abaqus 2022/Python interface.

The automation system consisted of two main parts. First, one file contained all of the data required to define a model (inputs.py).  This included the stalk CAD file path, material constants, boundary condition configuration, and others. Second, several functions processed the information in inputs.py to create the geometry, boundary conditions, meshes, and analysis. There were eight main subroutines for creating and analyzing maize stalk geometry. The end result was an automated package that can structurally analyze any specified stalk geometry at the push of a button, with complete control over material properties, boundary conditions, and analysis type. A diagram of the code functions that constitute the automation system is provided in ~\cref{fig:CodeDiagramSimplified}. The following sections detail the purpose and functionality of each subroutine.

\begin{figure}[htbp]
	\centering
	\includegraphics[scale=1.05]{figures/CodeDiagramSimplified}
	\caption[Code automation flowchart.]{Code automation flowchart. This is a simplified version of the actual code. The diagram above shows the general layout that aligns with the sections below.}
	\label{fig:CodeDiagramSimplified}
\end{figure}

\subsection{Subroutine 1: Importing .STEP files}
\label{ssec:subroutine_1_importing_step_files}
The first subroutine imported a pre-generated stalk geometry into Abaqus 2022. This geometry was manipulated in later subroutines to have user-specified boundary conditions, constraints and material constants for simulation purposes. In addition, this subroutine added guiding lines to the geometry for easier meshing. These lines were drawn based on coordinate axes.~\cref{fig:ImportSTEP} shows examples of the imported .STEP files. 

\begin{figure}[htbp]
	\centering
	\includegraphics[scale=1.05]{figures/ImportSTEP}
	\caption[STEP files]{STEP files. Pith on the left, rind on the right.}
	\label{fig:ImportSTEP}
\end{figure}

\subsection{Subroutine 2: Interactions}
\label{ssec:subroutine_2_interactions}
A tie constraint relationship was used to characterize the interaction between pith and rind. A tie constraint restricts the movement of the secondary surface, ensuring that it matches the displacement of the primary surface. Because the rind controls the majority of stalk behavior, the inner rind was defined as the primary surface, and the outer pith was defined as the secondary surface.

\subsection{Subroutine 3: Defining Material Constants}
\label{ssec:subroutine_3_defining_material_constants}
The third subroutine defined and applied material constant definitions to the imported geometry. Both the pith and rind were modeled as transverse isotropic materials. A transverse isotropic material has six material constants, five of which are independent.~\cref{fig:TransverseIsotropy_Angled} illustrates these six material constants, which are also listed in~\cref{tab:material_ranges}.
\newline

\begin{figure}[htbp]
	\centering
	\includegraphics[scale=0.95]{figures/TransverseIsotropy_Angled}
	\caption[Interpretation of transverse isotropy material constants.]{Interpretation of transverse isotropy material constants. The || symbol indicates a property in the fiber direction, while the ${\perp}$ symbol indicates a property perpendicular to the fibers. Material direction is highlighted for the rind tissue, but this orientation is also present in pith tissue.}
	\label{fig:TransverseIsotropy_Angled}
\end{figure}

To accurately model the material response of maize stalks, twelve different material constants were required, ten of which were independent.~\cref{tab:material_ranges} lists the ranges used in this study for each of these material constants along with their sources.

\begin{table}[htbp]
	\centering
	\caption[Material constant ranges for maize stalk pith and rind tissue.]{Material constant ranges for maize stalk pith and rind tissue, from Ottesen~\protect\cite{ottesen_development_2023-1}. Units in GPa.}
	\label{tab:material_ranges}	
    \renewcommand{\arraystretch}{1.75} % General row spacing

    \begin{tabular}{|>{\centering\arraybackslash}m{1.35cm}|>{\centering\arraybackslash}m{1.65cm}|>{\centering\arraybackslash}m{1.9cm}|>{\centering\arraybackslash}m{2cm}|>{\centering\arraybackslash}m{2cm}|>{\centering\arraybackslash}m{2.35cm}|}
		\hline
		\textbf{property} & \textbf{method} & \textbf{distribution} & \textbf{pith} & \textbf{rind} & \textbf{source} \\
		\hline
		${E_{\perp}}$ & random sampling & normal & (0.026, 0.01) & (0.85, 0.39) & Stubbs~\protect\cite{stubbs_maize_2022} \\
        \hline
		rind ${E_{||}}$ & specimen specific & empirical & n/a & specimen specific & Al-Zube~\protect\cite{al-zube_elastic_2018} \\
        \hline
		pith ${E_{||}}$ & random sampling & normal & (0.45, 0.05) & n/a & Sutherland~\protect\cite{sutherland_influence_2022} \\
        \hline
		${G_{\perp}}$ & calculated & n/a & \multicolumn{2}{c|}{\text{\Large ${\frac{E_{\perp}}{2(1+\nu)}}$}} & theory \\
        \hline
        \renewcommand{\arraystretch}{1.5}
		${G_{||}}$ & random sampling & normal & (0.27, 0.01) & (0.93, 0.33) & Carter~\protect\cite{carter_measurement_2023} \\
        \hline
		${\nu_{\perp}}$ & random sampling & uniform & \multicolumn{2}{c|}{(0.2, 0.45)} & \multirow{2}{2.5cm}{\centering Green~\protect\cite{green_wood_1999}} \\
        \cline{1-5}
		${\nu_{||}}$ & random sampling & uniform & \multicolumn{2}{c|}{(0.009, 0.086)} & \\
		\hline
	\end{tabular}
\end{table}

Due to uncertainty in several properties, a variance test was performed to determine relative influence of each material constant. This involved testing 63 stalks with 10 replicates per stalk. Each replicate was randomly sampled within the bounds listed in~\cref{tab:material_ranges}, except for ${E_{||}}$ in the rind, which was taken from stalk-specific physical testing~\scite{al-zube_measuring_2017}. The results of this test are discussed further in~\cref{ssec:validation_results}.

\subsection{Subroutine 4: Reference Points}
\label{ssec:subroutine_4_reference_points}
Specific reference points were added to each stalk on the vertices shown in~\cref{fig:ReferencePoints}. A tie constraint was used for reference points on the end faces, while a rigid body constraint with radii of effect was applied to the reference points in the nodal region. These interactions best mimic physical testing conditions. These reference points are used to easily apply loads and boundary conditions, which are discussed in~\cref{ssec:subroutine_5_loading_cases}.
\newline
\newline

\begin{figure}[htbp]
	\centering
	\includegraphics[scale=0.75]{figures/ReferencePoints}
	\caption[Reference point locations on maize stalk geometries.]{Reference point locations on maize stalk geometries.}
	\label{fig:ReferencePoints}
\end{figure}

\subsection{Subroutine 5: Loading Cases}
\label{ssec:subroutine_5_loading_cases}
One of the main goals of this study was to mimic the same conditions used to test the strength of physical maize stalks. This way, the results of simulated maize stalks (in FEA) could be compared against physical maize stalks for validation. This increases our confidence in the parameterized model.

The physical stalk responses used for this study were ones tested by Al-Zube~\scite{al-zube_measuring_2017,al-zube_elastic_2018}. Al-Zube tested specimens under three point bending loads. Three point bending loads were applied to each model.~\cref{tab:load_conditions} summarizes these boundary conditions.

\begin{table}[htbp]
    \centering
    \caption[Boundary condition calculations.]{Boundary condition calculations, from Ottesen~\protect\cite{ottesen_development_2023-1}.}
    \label{tab:load_conditions}    
    \renewcommand{\arraystretch}{1.65} % General row spacing

    \begin{tabular}{|>{\centering\arraybackslash}p{0.5in}|>{\centering\arraybackslash}p{1.85in}|>{\centering\arraybackslash}p{1.85in}|}
        \hline
        \textbf{Load Type} & \textbf{Bottom Face} & \textbf{Top Face} \\
        \hline
        Force & ${F_{a} = \frac{M_{fail}}{A}}$ & ${F_{b} = \frac{M_{fail}}{B}}$ \\
        \hline
        Moment & ${M_{a} = \frac{A - a}{A} M_{fail}}$ & ${M_{b} = \frac{B - b}{B} M_{fail}}$ \\
        \hline 
        \multicolumn{3}{|c|}{\rule{0pt}{1.1in}\includegraphics[scale=0.65]{figures/LoadDiagram}} \\
        \hline
        A & \multicolumn{2}{>{\raggedright\arraybackslash}p{4in}|}{Distance between left-hand support and applied load in the 3-point bending test} \\
        \hline
        B & \multicolumn{2}{>{\raggedright\arraybackslash}p{4in}|}{Distance between the right-hand support and applied load in the 3-point bending test} \\
        \hline
        ${M_{fail}}$ & \multicolumn{2}{>{\raggedright\arraybackslash}p{4in}|}{Maximum bending moment applied during physical 3-point bending tests} \\
        \hline
    \end{tabular}
\end{table}

For modeling purposes, only the middle of each stalk was considered in calculations. “A” was the length of the top half of each stalk, “B” was the length of the bottom half of each stalk, and “a” or “b” were the \textit{modeled} section of each of these sections. These test conditions are also shown in ~\cref{fig:ShearMomentDiagrams}.

\begin{figure}[htbp]
	\centering
	\includegraphics[scale=1.05]{figures/ShearMomentDiagrams}
	\caption[Load diagrams for three point bending of maize stalks.]{Load diagrams for three point bending of maize stalks. Writing in terms of a normalized Pfail yields the equations shown. Loads were normalized by Pfail because the linear buckling analysis requires a scalable input load.}
	\label{fig:ShearMomentDiagrams}
\end{figure}

This subroutine applied these loads and moments to the reference points described in~\cref{ssec:subroutine_4_reference_points} and constrained the middle section of each stalk. As will be discussed in~\cref{ch:ch4}, there were also other load configurations applied to each model, but for validation, the load conditions described above were used.

\subsection{Subroutine 6: Symmetry}
\label{ssec:subroutine_6_symmetry}
To handle potentially long and repetitive tests, simplifying the geometry can significantly reduce processing time. One effective method is to apply symmetry conditions. Because the parameterized maize stalk model has an elliptical cross-section, it has two symmetry planes. A symmetry boundary condition in the yz plane was used to simplify the model and reduce computational expense. The model was further simplified based on the fact that maize stalk failure typically occurs above the node~\scite{robertson_maize_2016}. This means that it was unnecessary to simulate material that is below the stalk node. Therefore, simulations only modeled above the node, across the yz plane; these models were called quarter models.~\cref{fig:GeometrySimplifications} summarizes the geometry simplifications applied to each stalk geometry.
\newline
\newline

\begin{figure}[htbp]
	\centering
	\includegraphics[scale=1.05]{figures/GeometrySimplifications}
	\caption[Simplifications to stalk geometry.]{Simplifications to stalk geometry.}
	\label{fig:GeometrySimplifications}
\end{figure}

\subsection{Subroutine 7: Mesh}
\label{ssec:subroutine_7_mesh}
The mesh used for each stalk consisted of two parts: a C3D4 (quadratic) tetrahedral region and a C3D8 hexahedral region.~\cref{fig:Mesh} shows the layout of the mesh applied to each model geometry. 
\newline

\begin{figure}[htbp]
	\centering
	\includegraphics[scale=1.05]{figures/Mesh}
	\caption[Mesh regions in maize stalk models.]{Mesh regions in maize stalk models, from Ottesen~\protect\cite{ottesen_development_2023-1}.}
	\label{fig:Mesh}
\end{figure}

In general, tetrahedral (tet) elements have more weaknesses than hexahedral (hex) elements, and hex elements should be used whenever possible~\scite{benzley_comparison_1995}. Tet elements were used in the node region due to significant geometric variation that hex elements cannot accurately capture. Hex elements were used in the internode section of the geometry.

The mesh density was adjusted by assigning specific edges a designated number of elements, calculated as a multiple of a model mesh index parameter stored in inputs.py. Increasing the model mesh index parameter enhances the mesh density along each feature edge. A custom multiplier for each feature edge was determined iteratively. For simplicity, this thesis does not detail the specific number of elements along each feature edge.

\subsection{Subroutine 8: Analysis and Post Processing}
\label{ssec:subroutine_8_analysis_and_post_processing}
The final subroutine ran a user-selected analysis and then post-processed the results of the analysis. Subroutine 8 was therefore composed of three ‘mini subroutines’ that all work toward the goal of running analysis and interpreting results: 1) running the chosen analysis, 2) flex post processing, and 3) buckling post processing. If the user chose to run a flex analysis, then only functions 1) and 2) were used; if the user chose to run a buckling analysis, then only functions 1) and 3) were used. 

\subsubsection{Running the Chosen Analysis}
\label{sssec:running_the_chosen_analysis}
Two different analyses could be performed on stalk geometries: linear static analysis and linear buckling analysis. These analyses were chosen because they most accurately capture the behavior observed from two different tests performed on physical maize stalks: flexural stiffness and failure strength. Physical testing was performed in three point bending configuration~\scite{al-zube_measuring_2017,al-zube_elastic_2018}.

Flexural stiffness refers to how stiff a stalk is; it describes how much force a stalk exerts when bent in the linear elastic region (when failure or nonlinearities are not present). This behavior can be replicated in FEA by applying a small load on a stalk and calculating its deflection. For small loads, a linear static (Newton) finite element solver is suitable for this situation.

Failure strength refers to the maximum applied force on a stalk before it fails. A linear eigenvalue (buckling) solver was used to calculate the maximum load before failure for stalks in three point bending. A linear eigenvalue solver operates by slowly scaling input loads until instability occurs. As such, the solver does not take into consideration any dynamic effects or material failure in the analysis. In spite of this, linear buckling analysis has been a commonly accepted and validated approach for calculating failure strength in maize stalk models~\scite{ottesen_parameterised_2022,ottesen_development_2023-1,stubbs_maize_2022}.

Subroutine 8 therefore ran either a flexural stiffness (linear static) or a failure strength (linear buckling) analysis depending on which analysis the user chose. This was as simple as defining a keyword (either ‘flex’ or ‘buckle’) to Abaqus to start the solver.

\subsubsection{Flex Post Processing}
\label{sssec:flex_post_processing}
The results of a linear static analysis were deflections caused by the applied loads. The flexural stiffness, {$EI$}, has been used in previous studies as a predictor of stalk strength~\cite{ottesen_development_2023-1,robertson_maize_2016}. Deflection was used with Castigliano’s theorem to calculate {$EI$}. Castigliano’s theorem is shown in~\cref{eq:deflection_equation,eq:castigliano_1,eq:castigliano_2}.

\begin{equation}
	\label{eq:deflection_equation}
	\delta = \frac{\partial}{\partial F} U
\end{equation}

\begin{equation}
	\label{eq:castigliano_1}
	U = \int \frac{M^{2}}{2EI} dx
\end{equation}

\begin{equation}
	\label{eq:castigliano_2}
	\delta = \int \frac{\partial M}{\partial F} \frac{M}{EI} dx
\end{equation}

Flexural stiffness was then calculated using equations from~\cref{tab:load_conditions} (see~\cref{eq:castigliano_3,eq:castigliano_4}). Subroutine 8 performed these calculations on calculated deflections and returned the flexural stiffness to the user in CSV format.

\begin{equation}
	\label{eq:castigliano_3}
	\delta_{a} = \frac{1}{2} \int_{0}^{a} \frac{2M_{a}+Px}{EI} dx
\end{equation}

\begin{equation}
	\label{eq:castigliano_4}
	EI_{a} = \frac{1}{\delta_{a}}(\frac{M_{a} a^{2}}{2} + \frac{Pa^{3}}{3})
\end{equation}

\subsubsection{Buckling Post Processing}
\label{sssec:buckling_post_processing}
Linear buckling analysis produced eigenvalues and corresponding eigenvectors. The eigenvalues are scale factors; multiplying the user-defined applied load (see~\cref{tab:load_conditions}) by a buckling eigenvalue corresponds to the load required to cause structural instability. The eigenvectors are the shape of the instabilities (sometimes called a mode).

During linear buckling analysis, there was a chance that a computed failure strength corresponded to a buckling shape that was unrealistic.~\cref{fig:BucklingProblems} illustrates common shapes that are unrealistic, such as ones influenced by mesh node locking or boundary effects.

\begin{figure}[htbp]
	\centering
	\includegraphics[scale=0.85]{figures/BucklingProblems}
	\caption[Examples of node locking, boundary effects, and correct modes from linear buckling analysis.]{Examples of node locking, boundary effects, and correct modes from linear buckling analysis.}
	\label{fig:BucklingProblems}
\end{figure}

To classify the computed buckling nodes, a method was developed to check a computed shape for inaccuracies. This involved (1) exporting the buckling shape as a set of four contour lines, (2) determining which contour line captures the buckling shape, and (3) calculating the number of peaks and valleys of the contour line of interest (see~\cref{fig:EigenProcess}). This caught buckling shapes calculated below the node and shapes affected by mesh node locking (shapes with more than 20 peaks and valleys). Subroutine 8 returned the eigenmode, eigenvector, and number of peaks and valleys to the user in CSV format. A minimum of three buckling shapes were calculated for each failure strength analysis with the knowledge that some computed shapes would be unrealistic.

\begin{figure}[htbp]
	\centering
	\includegraphics[scale=1.05]{figures/EigenProcess}
	\caption[Post processing linear buckling results.]{Post processing linear buckling results.}
	\label{fig:EigenProcess}
\end{figure}

\subsection{Subroutine Summary}
\label{ssec:subroutine_summary}
The culmination of the eight subroutines described above is an automated package capable of generating and analyzing 3D parameterized maize stalk models. This method significantly reduces processing time compared to CT-based methods and facilitates in-depth studies on geometric sensitivity, as detailed in~\cref{ch:ch4}.

\section{System Validation}
\label{sec:system_validation}

\subsection{Comparison of Quarter Symmetric Models to Whole Models}
\label{ssec:comparison_of_quarter_symmetric_models}
To verify that quarter symmetry simplifications did not significantly influence the predictive accuracy of our models, simulations were conducted on twenty stalk geometries using the quarter model simplification and compared to simulations of the full geometry. The results from both sets of models showed a high correlation, with an ${r^{2}}$ statistic of over 0.99 for both flexural stiffness and failure strength analyses.
\newline
\newline

\subsection{Mesh Convergence}
\label{ssec:mesh_convergence}
An important aspect of FEA meshing is the mesh convergence study~\scite{hughes_finite_1987}. This involves increasing mesh density at discrete intervals to evaluate the mesh's influence on results. For our geometries, a mesh convergence study was performed on both the pith and rind geometries by varying the model mesh index parameter. 

The mesh convergence study consisted of increasing the model mesh index parameter in steps of 0.5. For pith convergence, the rind was kept at a fine model mesh index of 7 and the pith model mesh index was slowly increased from a coarse mesh of 1. The reverse was true for rind convergence. \cref{fig:MeshConvergence} shows the mesh convergence of the pith and rind geometries for both flexural stiffness analysis and linear buckling analysis. Convergence tests were conducted for both whole and quarter geometries, with nearly identical results. As seen in~\cref{fig:MeshConvergence}, an adequate model mesh index for both pith and rind meshes was a mesh index parameter of 3. Summary statistics for the number of elements used in the pith and rind regions, as well as the type of element with a model mesh index of 3, are provided in~\cref{sec:assessment_of_automation_system_performance}.

\begin{figure}[htbp]
	\centering
	\includegraphics[scale=1.05]{figures/MeshConvergence}
	\caption[Mesh convergence.]{Mesh convergence. The model mesh index represents the relative degree of how fine the mesh is; 1 is a coarse mesh and 7 is a fine mesh. The y-axis shows the percent change from the last calculated result. For example, a percent change of 17 for pith buckling convergence at a model mesh index of 1 means the calculated buckling result at a model mesh index of 1 was 17\% higher than the same result calculated at a model mesh index of 0.5. After a model mesh index of 3, the percent change between subsequent indices was less than 1\%.}
	\label{fig:MeshConvergence}
\end{figure}

\subsection{Validation Results}
\label{ssec:validation_results}
To validate the automation system, 63 maize stalks were generated and analyzed using Abaqus 2022, incorporating the methodologies outlined in the previous sections. Flexural stiffness (flex) and failure strength (buckling) test results were compared with physical testing results. As described in~\cref{ssec:subroutine_3_defining_material_constants}, variance was introduced to account for material uncertainties. The test results are shown in~\cref{fig:FlexValidation,fig:BucklingValidation}.

\begin{figure}[htbp]
	\centering
	\includegraphics[scale=1.05]{figures/FlexValidation}
	\caption[Whole geometry material variance and validation for flexural stiffness.]{Whole geometry material variance and validation for flexural stiffness, from Ottesen~\protect\cite{ottesen_development_2023-1}.}
	\label{fig:FlexValidation}
\end{figure}

\begin{figure}[htbp]
	\centering
	\includegraphics[scale=1.05]{figures/BucklingValidation}
	\caption[Whole geometry material variance and validation for failure strength.]{Whole geometry material variance and validation for failure strength, from Ottesen~\protect\cite{ottesen_development_2023-1}.}
	\label{fig:BucklingValidation}
\end{figure}

The 3D parameterized maize stalk model demonstrated strong correlations between model predictions and actual maize stalk behavior, particularly in flexural stiffness calculations. The strength of the relationship was not as strong for failure strength calculations, but still predictive of stalk strength. Buckling analyses are more sensitive to material constants as a result of the more complex mechanics of buckling. This issue is discussed further in~\cref{ch:ch4}.

\section{Assessment of Automation System Performance}
\label{sec:assessment_of_automation_system_performance}
While model creation would typically require manual execution by an experienced finite element analyst, the automated approach demonstrated in this chapter drastically reduces setup time. Manual execution typically takes at least 20 minutes per analysis, whereas the automated method presented here completes the same tasks in just seconds.~\cref{tab:generation_time} summarizes the mean model generation time for both whole and quarter models, for models with a model seed index of 3 (see~\cref{ssec:mesh_convergence}). These statistics were gathered by generating 20 models of each case (20 quarter models, and 20 whole models). The 20 models were chosen through failure strength stratified sampling of the original 900 CT scanned stalks.

\begin{table}[htbp]
    \centering
    \caption[Mean model generation time for whole and quarter models.]{Mean model generation time for whole and quarter models, in seconds.}
    \label{tab:generation_time}    
    \renewcommand{\arraystretch}{1.25} % General row spacing

    \begin{tabular}{|>{\centering\arraybackslash}p{4.15cm}|>{\centering\arraybackslash}p{1.5cm}|>{\centering\arraybackslash}p{1.5cm}|>{\centering\arraybackslash}p{3cm}|}
        \hline
        & \textbf{Whole} & \textbf{Quarter} & \textbf{Whole / Quarter} \\
        \hline
        \textbf{Model Generation Time} & 15 s & 9 s & 1.67 \\
        \hline
    \end{tabular}
\end{table}

\cref{tab:analysis_time} shows the mean analysis runtime of these same generated models for both flex and buckling simulations. Quarter models run roughly twice as fast for flex simulations and nearly six times as fast for buckling simulations, while providing virtually identical results. 

\begin{table}[htbp]
    \centering
    \caption[Mean runtime for flex/buckling models for whole/quarter models.]{Mean runtime for flex/buckling models for whole/quarter models, in seconds.}
    \label{tab:analysis_time}    
    \renewcommand{\arraystretch}{1.25} % General row spacing

    \begin{tabular}{|>{\centering\arraybackslash}p{2cm}|>{\centering\arraybackslash}p{1.5cm}|>{\centering\arraybackslash}p{1.5cm}|>{\centering\arraybackslash}p{3cm}|}
        \hline
        & \textbf{Whole} & \textbf{Quarter} & \textbf{Whole / Quarter} \\
        \hline
        \textbf{Flex} & 47 s & 25 s & 1.88 \\
        \hline
        \textbf{Buckling} & 471 s & 81 s & 5.81 \\
        \hline
    \end{tabular}
\end{table}

\cref{tab:mesh_characteristics} shows the average and standard deviation number of elements present in both whole and quarter models for both pith and rind regions for each element type. This data was taken from the same 20 models described previously.

\begin{table}[htbp]
    \centering
    \caption[Mean/standard deviation number of elements in each region by element type.]{Mean/standard deviation number of elements in each region by element type, also in percent form.}
    \label{tab:mesh_characteristics}    
    \renewcommand{\arraystretch}{1.25} % General row spacing

    \begin{tabular}{|>{\centering\arraybackslash}p{0.85in}|>{\centering\arraybackslash}p{0.8in}|>{\centering\arraybackslash}p{1in}|>{\centering\arraybackslash}p{0.8in}|>{\centering\arraybackslash}p{1in}|}
        \hline
        \textbf{Element} & \multicolumn{2}{c|}{\textbf{Whole Models}} & \multicolumn{2}{c|}{\textbf{Quarter Models}} \\
        \hline
        & \textbf{mean (std)} & \textbf{\%} & \textbf{mean (std)} & \textbf{\%} \\
        \hline
        tet (rind) & 38207 (6817) & 44.09\% (7.87\%) & 0 (0) & 0.00\% (0.00\%) \\
        \hline
        hex (rind) & 10752 (0) & 12.41\% (0.00\%) & 2800 (0) & 16.43\% (0.00\%) \\
        \hline
        wedge (rind) & 0 (0) & 0.00\% (0.00\%) & 600 (118) & 3.52\% (0.69\%) \\
        \hline
        tet (pith) & 13310 (746) & 15.36\% (0.86\%) & 6140 (911) & 36.03\% (5.35\%) \\
        \hline
        hex (pith) & 24384 (1160) & 28.14\% (1.34\%) & 7500 (0) & 44.01\% (0.00\%) \\
        \hline
        wedge (pith) & 0 (0) & 0.00\% (0.00\%) & 0 (0) & 0.00\% (0.00\%) \\
        \hline
    \end{tabular}
\end{table}

Note that wedge elements were specifically used to replace tetrahedral elements in the rind region for quarter models. This was only feasible for quarter models, and was done because wedge elements perform slightly better than tet elements. The use of wedge elements only slightly affected convergence results in~\cref{ssec:mesh_convergence}, which was performed for both whole and quarter models with similar results.

For hexahedral element regions, the standard deviation for both pith and rind meshes was zero (except for whole model pith hex elements, which were minimal), indicating consistent meshing regardless of geometry. In contrast, tetrahedral and wedge element regions showed non-zero standard deviations for both pith and rind meshes, reflecting some numerical noise associated with geometric changes in those regions.

\section{Discussion}
\label{sec:automation_discussion}
The results of the work in this chapter is a fully automated code architecture capable of analyzing any 3D parameterized maize stalk geometry. This allows for researchers to more freely explore model behavior in sensitivity studies. The automated nature of this code architecture facilitates rapid analysis of arbitrary model shapes, significantly enhancing efficiency in computational studies.

Validation results depicted in~\cref{fig:FlexValidation,fig:BucklingValidation} demonstrate the predictive capability of the automated package.  This makes the model a reliable tool for future investigations into maize stalk mechanics and behavior. It is estimated that over 50,000 simulations have been run using this automated package over the past 3 years.

The outcomes of this chapter lays a solid foundation for subsequent chapters, where the automated code architecture will play a pivotal role in parameter sensitivity analysis. This automated architecture can also be used in future studies involving optimization of geometric parameters. 

