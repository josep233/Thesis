\chapter{Summary and Conclusion}
\label{ch:ch5}

\section{Summary of Results}
\label{sec:summary_of_results}
The purpose of this study was to improve and better understand the factors that influence maize stalk failure. First, the process for creating and analyzing 3D parameterized finite-element models of maize stalks was automated. Second, the longitudinal shear modulus of maize stalk pith and rind tissues were measured to fill a gap in our current set of material constants used in FEA analysis. Third, a sensitivity analysis was performed to assess the influence of material and geometric factors on maize stalk flexural stiffness, failure strength, and biomass.

The creation and analysis of 3D parameterized models in FEA were automated using the python/Abaqus API, which allows for complete user control over boundary conditions, meshes, and geometries. The created automated package can now create a model in under one minute. This is a major improvement over the 20 minutes previously required when manually creating a model. Validation of the automated package showed high correlation between simulation results and results from physical testing, with an ${r^{2}}$ of 0.98 for flexural stiffness, and an ${r^{2}}$ of 0.73 for failure strength~\scite{ottesen_development_2023-1}. The automated modeling platform supports the automated creation of models that can be used for sensitivity analyses and optimization studies.

The transverse shear modulus of maize pith and rind tissue were measured for dry and fully mature maize stalks. The shear modulus of the rind tissue had a mean of 931 MPa and standard deviation of 334 MPa. The pith shear modulus had a mean of 27 MPa and standard deviation of 10 MPa. Rind measurements had similar values to bamboo and wood. These measurements are the first reported values in the literature. Their use in FEA models allows researchers to avoid the estimation process that was necessary before these measurements were taken. 

The automated model generation package was used to perform a sensitivity analysis in order to assess the influence of material and geometric factors on model flexural stiffness, failure strength, and biomass. Sensitivity analysis consisted of making changes to statistical geometric patterns found in actual maize. These statistical geometric patterns were found using principal component analysis. Manipulating the principal component patterns increased the ‘amount’ of a particular pattern found in actual maize. Results indicated high failure strength sensitivity to the first principal component, with low sensitivity to all other principal components. The first principal component is highly correlated with section modulus (${r^{2}=0.97}$);  this is independent validation that section modulus is the primary predictor of stalk strength, which is consistent with previous studies~\scite{robertson_maize_2017}. Results indicated that all failure strength sensitivities had positive correlations with biomass. Because there is only one influential predictor of stalk strength, and because all other factors lead to increases in biomass, the efficacy of biomass-constrained optimization is unlikely. This is an important conclusion that informs proposed solutions to stalk lodging.

\section{Contributions}
\label{sec:contributions_to_scientific_literature}
The automated system described in~\cref{ch:ch2} supported a study focused on validating the parameterized maize stalk model.  The validation study was published in the journal \textit{Plant Methods} and I was a co-author on that paper~\scite{ottesen_development_2023-1}. The automated system  also contributed to the material and geometric sensitivity analysis in~\cref{ch:ch4}.

The measurement of longitudinal shear modulus of pith and rind tissue in~\cref{ch:ch3} resulted in a paper that has been accepted for publication in the journal \textit{Plant Methods}~\scite{carter_measurement_2023}, These results were subsequently used in the models described in~\cref{ch:ch4}.

The sensitivity analysis in~\cref{ch:ch4} will soon be submitted to the journal \textit{in Silico Plants}. These results will also inform future studies concerning the effect of geometric parameters on stalk response.

\section{Future Work}
\label{sec:future_work}
Several future studies would be beneficial to complement the work done in this thesis. Firstly, now that a solid relationship has been established between principal components and model response, optimization studies could be performed to improve model failure strength by manipulating the principal components rather than individual model parameters. Based on the results of the sensitivity analysis, it seems unlikely that model failure strength can be increased significantly without increasing biomass. This is because the sensitivity analysis revealed only positive correlations between failure strength and biomass. However, an optimization study that does not constrain biomass could quantify the degree of biomass increase; it may be that the tradeoff between increasing stalk biomass and increasing stalk strength is small enough to warrant attention in selective breeding.

Second, if stalk strength cannot be feasibly increased with additional biomass, future studies will need to focus on increasing stalk strength in other ways. While this thesis sought to preserve variation patterns found in nature (through principal component analysis), it may be the case that breaking these patterns may allow for increases in failure strength without significantly increasing biomass. The downside to this approach is that it may indicate an optimal solution that is difficult or impossible to achieve through natural breeding.

In addition to optimization studies, further work could focus on some factors that are currently ignored in our models. For example, our models only model a small portion of maize stalks (about 50 millimeters above and below a single node) instead of the entire stalk. In addition, our models do not include the leaf sheath, which has been found to be influential in stalk failure~\scite{ogilvie_effects_2024}. Lastly, our models do not include the relationship between stalks and the soil. The addition of each of these factors would lead to a more comprehensive model. Fortunately, the current model provides a flexible platform upon which these additional capabilities could be built. Thus it is hoped that this model will both be used in future studies in its current form and expanded upon to increase its capabilities. 

\section{Conclusions}
\label{sec:conclusions}
The results of this thesis are 1) an automated modeling package that allows models of any desired shape and material properties to be created approximately 100 times faster than previously possible; 2) distributions of longitudinal shear modulus of pith and rind tissue, found to be 13 MPa to 55 MPa for pith tissue and 355 MPa to 1630 MPa for rind tissue; and 3) evidence that the strength and flexibility of maize stalks are governed by a single mode of geometric variation, which is positively correlated with biomass and closely related to section modulus. These findings significantly contribute to the field of engineering in agronomy, and help researchers better understand the factors that contribute to maize stalk lodging.